\documentclass[english,12pt,letterpaper]{article}
\usepackage[utf8]{inputenc}
\usepackage[T1]{fontenc}
\usepackage[width=8.50in, height=11.00in, left=0.50in, right=0.50in, top=1in, bottom=1in]{geometry}
\usepackage{babel}
\author{KC1AWV}
\begin{document}
	\begin{center}
		\section*{Constitution}
	\end{center}
	\section*{Preamble}
	We, the undersigned, wishing to secure for ourselves the pleasures and benefits of an association of persons commonly interested in Amateur Radio, constitute ourselves the M17 Amateur Radio Club of the United States and enact this constitution as our governing law. It shall be our purpose to further the exchange of information and cooperation between members, to promote radio knowledge, fraternalism and individual operating efficiency, and to so conduct club programs and activities as to advance the general interest and welfare of Amateur Radio in the community.
	\section{Article I - Membership}
	All persons interested in Amateur Radio communications shall be eligible for membership. Membership shall be by application and election upon such terms as the club shall provide in its By-Laws. \\
	\\
	Membership may not be denied because of race, creed, color, religion, gender, sexual orientation, political affiliation, marital status or any other reason that would be biased or prejudicial.
	\section{Article II - Officers}
	The officers of this club shall be President, Vice-President, Secretary and Treasurer.
	\subsection{Election}
	The officers of this club shall be elected for a term of one year by ballot of the members present, provided there be a quorum, at the annual meeting.
	\subsection{Term Limits}
	An individual may not hold the same office for more than two terms and my not serve more than three consecutive terms as an officer. \\
	\\
	An individual may not hold more than one office during the same term.
	\subsection{Vacancies}
	Vacancies occurring between elections must be filled by special elections at the first regular meeting following the withdrawal or resignation.
	\subsection{Eligibility}
	In order to hold an office an individual must be a member in good standing for at least one year and hold a valid Amateur Radio license.
	\subsection{Resignation}
	Any officer may resign his/her position in writing at which time all records and assets of the club will be turned over to the President or Vice President.
	\subsection{Removal of Officers}
	Officers may be removed from office for cause, upon written petition of six (6) or more members presented to the President or Vice President. \\
	\\
	After investigation the petition will be presented to the membership at the next regular meeting of the club and a voted on by the membership. \\
	\\
	Removal of an officer requires a three-fourths vote of the full membership.
	\section{Article III - Duties of Officers}
	\subsection{President}
	The President shall preside at all meetings, and conduct them according to the rules adopted. He/she shall enforce due observance of this Constitution and By-Laws; decide all questions of order; sign all official documents adopted by the club, and perform all other duties pertaining to the office of President.
	\subsection{Vice President}
	The Vice-President shall assume all the duties of the President in his/her absence. In addition, he/she shall organize club activities, plan and recommend contests for operating benefits, and advance club interest and activity as approved by the club. He/she shall maintain close liaison with the local ARRL's local ARES® Emergency Coordinator to further club participation in the Amateur Radio Emergency Service.®
	\subsection{Secretary}
	The Secretary shall keep a record of the proceedings of all meetings, keep a roll of members, submit membership applications, carry on all correspondence, read communications at each meeting, and mail written meeting notices to each member. At the expiration of his/her term he/she shall turn over all items belonging to the club to his/her successor.
	\subsection{Treasurer}
	The Treasurer shall receive and receipt for all monies paid to the club; keep an accurate account of	all monies received and expended; pay no bills without proper authorization (by the club or its officers constituting a business committee). At the end of each quarter he/she shall submit an itemized statement of disbursements and receipts. At the end of his/her term he/she shall turn over everything in his/her possession belonging to the club to his/her successor.
	\section{Article IV - Meetings}
	The By-Laws shall provide for regular and special meetings. At meetings, a minimum of one-third of the membership shall constitute a quorum for the transaction of business. \\
	\\
	Robert's Rules of Order shall govern proceedings.
	\section{Article V - Dues}
	The club, by majority vote of those present at any regular meeting, may levy upon the general membership such dues or assessments as shall be deemed necessary for the business of the organization. Non-payment of such dues or assessments shall be cause for expulsion from the club within the discretion of the membership.
	\section{Article VI - Membership Assistance}
	The club, through designated interference, Public Relations, and Operating Committees will provide technical advice to members concerning equipment design and operation to assist in frequency observance, clean signals, uniform practice, and absence of spurious radiation's from club member-stations. The club shall also maintain a program to foster and guide public relations.
	\section{Article VII - Club Call Sign}
	The club may elect to apply for a club call sign as provided by FCC rules Part 97. \\
	\\
	The President shall assign trusteeship of the club call sign. \\
	\\
	The trustee shall: \\
	\begin{itemize}
		\item Be a member of the club in good standing
		\item Meet FCC requirements
		\item Not had his/her Radio Amateur licensed revoked or sanctioned at any time.
	\end{itemize}
	\section{Article VIII - Dissolution of the Club}
	\subsection{Termination of Operations}
	In the event that the Board of Director votes that the Club should be dissolved the motion for dissolution must receive more than two thirds vote of the full membership to pass.
	\subsection{Disposition of Assets}
	The Board of Directors shall handle the disbursement of all assets of the club. \\
	\\
	No member or group of members shall receive benefit from the assets. \\
	\\
	All equipment will be sold and net proceeds donated to a non-profit organization. \\
	\\
	All remaining cash will be donated to a non-profit organization.
	\section{Article IX - Amendments}
	This constitution or By-Laws may be amended by a two-thirds vote of the total membership. Proposals for amendments shall be submitted in writing at a regular meeting and shall be voted on at the next following regular meeting, provided all members have been noticed by mail of the intent to amend the constitution and/or By-Laws at said meeting.
\end{document}